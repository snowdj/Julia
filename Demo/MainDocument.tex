
\tableofcontents
%-----------------------------------------------------%
\section{Introduction}

\subsection{Background}

\subsection{Motivation}

\subsection{``Hello World"}

\subsection{Including Source Files}

You can also run files manually by using Julia's include command. This command can be used in the console or within other files. Including a file evaluates its contents and can be used to modularize your programs by splitting subsets of logic or libraries into individual files. In the console you have to give include the complete absolute path. 
%-----------------------------------------------------%
\newpage
\section{Basic Mathematical Operations}

\begin{framed}
\begin{verbatim}

\end{verbatim}
\end{framed}
%---------------------%
\subsection{Simple Matrix Operations}

\begin{framed}
\begin{verbatim}

\end{verbatim}
\end{framed}
%---------------------%
\subsection{Simnple Complex Arithmetic Operations}

\subsection{Generating Random Numbers}

\begin{itemize}
\item
\item
\item \texttt{srand}
\end{itemize}

%-----------------------------------------------------%
\section{Data Types}
Types in Julia are optional, but it's a good idea to specify types when it makes sense. In this case, the definition of a prime number implies that only natural numbers can be prime. It makes sense here to type the argument, n, to be an Integer. Julia has a few different Integer types, but the one you will use most often is Int64, unless you have a 32 bit computer, 
in which case your system will use Int32 by default.
\subsection{The \texttt{Float64} type }
\subsection{The \texttt{BigInt} Type}

\subsection{Type Conversion}
%-----------------------------------------------------%
\newpage
\section{Programming control Statements}
Loops are very simple with Julia. There are just two kinds, for loops and while loops.
\subsection{For Loops}

\subsection{While Loops}

\section{Other Statements}


%-----------------------------------------------------%
\newpage
\section{Writing Functions}
